%% ----------------------------------------------------------------
%% Project.tex
%% ----------------------------------------------------------------
\documentclass[openany]{ecsproject}     % Use the Project Style
\graphicspath{{images/}}   % Location of your graphics files
%\DeclareGraphicsExtensions{.png,.jpg}
\usepackage[square,numbers]{natbib}            % Use Natbib style for the refs.
\hypersetup{colorlinks=true}   % Set to false for black/white printing
\input{Definitions}            % Include your abbreviations
%\bibliographystyle{ecs}
\bibliographystyle{ieeetr}
%% ----------------------------------------------------------------
\begin{document}
\frontmatter
\title      {Energy efficient object detection and tracking through adaptive operation on heterogeneous multi-core systems}
\authors    {\texorpdfstring
             {\href{mailto:yz39g13@soton.ac.uk}{Yubo Zhi}}
             {Yubo Zhi}
            }
\addresses  {\groupname\\\deptname\\\univname}
\date       {\today}
\subject    {}
\keywords   {}
\supervisor {\texorpdfstring
             {\href{mailto:gvm@ecs.soton.ac.uk}{Dr Geoff Merrett}}
             {Yubo Zhi}
	    }
\examiner   {\texorpdfstring
             {\href{mailto:example@soton.ac.uk}{Someone}}
             {Someone}
	    }
\degree     {BEng Electronic Engineering}
\maketitle

\begin{abstract}
Object detection and tracking is a key technique used in computer vision applications, especially robotic systems, artificial intelligence and video surveillance systems, but video processing algorithms require a lot of computations, consume a significant amount of energy. This project was purposed to reduce power consumption and possibly improves performance of camera based real-time object detection and tracking applications. Such an application may rely heavily on battery power or harvesting energy from environment, hence power consumption on those devices becomes critical. As battery technology development was not as fast as electronic processing technology development, lower power consumption implies longer battery life for mobile devices e.g. laptops and smart phones, also lower environmental impact.

This project investigated some hardware based methodologies to achieve an overall reduction of computations required for doing continuous real-time object detection and tracking, includes dynamical control of hardware downsampling and cropping provided by the camera controller, automatic camera sleeping and frame rate reduction based on whether objects detected, their location and moving speed relative to the camera sense.

\iffalse
The implementations and investigations were hosted on the NVIDIA Tegra K1 embedded platform, features a GPU accelerated OpenCV library, .



In order to achieve this, automatic feedback control need to be developed based on previously recorded moving objects positions and velocities. GPU accelerated computer vision algorithm will also be deployed to minimise power consumption and computation time. For easier power consumption analysis, the Jetson embedded development platform featuring a quad-core ARM CPU and a CUDA-enabled Tegra GPU, introduced by NVIDIA, will be used in this project.

In technical aspect, OpenCV API for Tegra with CUDA GPU acceleration developed by NVIDIA for the embedded platform will be used to implement the video processing algorithm. The algorithm will employ the widely used background subtraction technique to detect foreground moving objects and get their coordinates. A camera will also be interfaced with the platform to do capturing and detecting objects in real-time, by a Linux kernel module driver. Afterwards, an automatic feedback frame rate and resolution control will be implemented, then power consumption with different operation modes will be investigated and the energy saved by applying the proposed method will be analysed.

If time allows, object identification may also be implemented, and possibly power saving technique will be investigated.

The aims of this project are:
	(internet enabled?)
Implement: // Using CUDA / OpenCV on Jetson
	Detect car in image
	Potentially track through series of images
Adaptive operation / optimise for energy: // Measurement & testing
	Improve energy efficiency by reduce frame rate when detected vehicle speed is slow
	Change image resolution
\fi
\end{abstract}

\tableofcontents
%\listoffigures
%\listoftables
%\lstlistoflistings
%\listofsymbols{ll}{$w$ & The weight vector}

%\acknowledgements{Thanks to no one.}
%\dedicatory{To \dots}
\mainmatter

%% ----------------------------------------------------------------
%% ----------------------------------------------------------------
%% Goal.tex
%% ----------------------------------------------------------------
\chapter{Project Goals} \label{Chapter:Goals}

\section{The problem}

With the rapid growth of integrated circuit manufacture technology, processing power was just enough to realise computation intensive computer vision algorithms by a small form factor mobile device. Such a battery powered mobile device provides considerable flexibility and is an essential part in robotic systems, artificial intelligence and machine learning. However, mobile devices generally have limited processing power, memory, storage and energy, but video processing algorithms involves lots of computations, hence could be the most power intensive and time consuming part in such application.

\section{Aims}

The aim of this project was to investigate ways to reduce overall power consumption of camera based real-time object detection and tracking applications, by applying feedback control of the camera module based on previous tracking results and utilising existing camera hardware capabilities including downsampling and cropping instead of software algorithms, while keeping a sensible accuracy. By doing so the average computation time would be dramatically reduced, hence reducing the overall power consumption.

The targets to be achieved are:

\begin{itemize}
  \item Development of a basic object tracking system, identify algorithms suitable for real-time operation, minimise operating system overheads.
  \item Interfacing camera module with the embedded system, expose control functions.
  \item Develop and analyse automatic camera frame rate reduction.
  \item Develop and analyse hardware downsampling and cropping.
  \item If time allows, object recognition may be implemented.
\end{itemize}

%% ----------------------------------------------------------------
%% Background.tex
%% ----------------------------------------------------------------
\chapter{Background and report of literature search}

There were lots of different algorithms existing for object detection and tracking. Some of those algorithms were investigated in this project in order to identify a set of suitable algorithms that were both accurate and efficient enough to analysis video stream from camera in real-time.

\section{Background subtraction}

Being able to detect objects in a video frame is the first, also the most difficult and important step to do object tracking. This is generally accomplished by separation of foreground objects and background image. Three different object detection methodologies were investigated in this project.

\subsection{Colour based}
\label{bgs:colour}

Colour can provides enough information of a specific object. For easier analysis of colour information, a hue-saturation-value (HSV) colourspace \cite[p.~301]{colourspace} representation converted from the original RGB colourspace is usually used, because it would be easier to filter a range of colour based on hue, saturation and brightness.

A simple colour based foreground mask can be generated easily by filtering target colour. For example, an simple implementation \cite{MOTBOC.git} based on colour filtering object detection, as shown in \fref{Figure:MOTBOC}.

\begin{figure}[H]
  \centering
  \includegraphics[width=0.6\columnwidth]{MOTBOC}
  \caption{Multi Object Tracking Based on Color (adapted from \cite{MOTBOC.git})}
  \label{Figure:MOTBOC}
\end{figure}

This implementation doesn't require a lot of computation, thus was very fast, could be suitable for robots that are tracking sonething like a single coloured ball or piece of paper, also could be useful for line racing car projects.

\subsection{Shape based}

Another important information about an object is its shape. By extracting hard object edges in the scene than apply appropriate shape transformation and filtering algorithms, an object could also be detected based on the shape.

\fref{Figure:circles} shows the image processed by circle detection, based on OpenCV's implementation of Hough Circle Transform \cite{opencv:hough_circle}. The frame captured from camera (\fref{Figure:edges:original}), was converted to gray scale and blurred first, as shown in \fref{Figure:edges:blur}. Smooth was necessarily to reduce possibly false circles that may be detected. Afterwards the object edges in the image was extracted as in \fref{Figure:edges:edges}. Finally \fref{Figure:edges:circles} shows the circles detected by the algorithm.

\begin{figure}[H]
  \centering
  \subfigure [] {
    \includegraphics[width=0.45\columnwidth]{simple_original}
    \label{Figure:edges:original}
  }
  \subfigure [] {
    \includegraphics[width=0.45\columnwidth]{simple_blur}
    \label{Figure:edges:blur}
  }
  \subfigure [] {
    \includegraphics[width=0.45\columnwidth]{simple_edges}
    \label{Figure:edges:edges}
  }
  \subfigure [] {
    \includegraphics[width=0.45\columnwidth]{simple_circles}
    \label{Figure:edges:circles}
  }
  \caption{Circle detection. \subref{Figure:edges:original} The original image, \subref{Figure:edges:blur} image converted to gray scale and blurred, \subref{Figure:edges:edges} edges detected, \subref{Figure:edges:circles} circles detected}
  \label{Figure:circles}
\end{figure}

This implementation could be suitable for ball tracking purpose. By combining with the simple colour filtering algorithm as described in Section \ref{bgs:colour}, a single coloured ball could be efficiently tracked.

\subsection{Cascade Classifier}

Cascade classifier \cite{cascade} is another widely used technique for object detection. It concatenates several classifiers detecting different object features to recognise objects, and it can be trained both positively and negatively to improve accuracy. It was usually used for object classify, for example recognise human and different classes of vehicles in a single frame.

The OpenCV's cascade classifier implementation \cite{opencv:cc} of face and eye detection was investigated as shown in \fref{Figure:cc_face}.

\begin{figure}[H]
  \centering
  \includegraphics[width=0.6\columnwidth]{"CC face"}
  \caption{Face and eye detection cascade classifiers, detected face was circled by pink, whereas detected eyes were circled by blue}
  \label{Figure:cc_face}
\end{figure}

\subsection{Motion based}

By differenceing current frame and previous frames then probably build up a background model image, it is also possible to detects moving objects efficiently. The BGSLibrary \cite{bgslibrary} is specifically developed for this purpose, it offers 37 different background substraction algorithms using OpenCV, published under GNU GPL v3 license.

This type of algorithms suit well for static camera movement tracking, therefore was used in this project.

\section{Movement tracking}

After obtained the foreground object mask, they need to be interpreted as objects, then the object can be detected and tracked based on its position.

\subsection{Connected component analysis} \label{blob}

Connected component analysis, or Connected component labeling, is used for detecting connected regions (blobs). A blob detector can be used to mark and labeling individual objects from the foreground mask, therefore obtain parameters such as size, position and orientation of the object. Afterwards, by comparing nearby objects from previous frames, the objects can be tracked, and movement parameter such as velocity and acceleration can then be obtained by physical modelling.

There were also lots of free and open source blob detection libraries available, e.g. the simple blob detector came with OpenCV \cite{opencv:blob}, cvBlob library \cite{cvblob}.

\subsection{Continuously Adaptive Meanshift}

Continuously Adaptive Meanshift (CAMshift) \cite{bradski1998computer} is a technique used to track a region of interest (ROI) in continuous frame sequences. It is based on Meanshift algorithm, which only track a fixed size ROI window, whereas CAMshift can handle target resize and rotation. In order to determine the ROI for CAMshift, the blob detector as described in Section \ref{blob} can also be used. This algorithm can also be easily implemented using OpenCV \cite{opencv:camshift}, and was used by lots of researches such as \cite{chu2007object}, \cite{xu2012moving} and \cite{nouar2006improved}.

%% ----------------------------------------------------------------
%% HWSW.tex
%% ----------------------------------------------------------------
\chapter{Platforms and interfaces}

This section describes the hardware platforms and software interfaces used to accomplish this project.

\section{Hardware platforms}

\subsection{Embedded host platform}

The Jetson Tegra K1 embedded development platform \cite{NVIDIA:tk1} featuring a 2.32GHz quad-core ARM CPU and a CUDA-enabled Tegra GPU, introduced by NVIDIA, was used in this project. The fact that it is an embedded platform enables easy power consumption analysis, a rich set of peripherals interfaces exposed enables direct control and interfacing a camera module and it is powerful enough to compile programs and execute complex computer vision algorithms.

\subsection{Camera module}
\label{hwsw:camera}

Specific camera module to be used was not determined yet at the time this progress report was written, but it would be a high resolution camera module from OmniVision \cite{ovt} that can be easily interfaced and directly controlled with the peripheral interfaces on the Jetson TK1 platform. A Linux kernel module driver probably need to be developed in order to control the camera parameters and operations from program running in user space.

\subsection{Testing platform}

The computer vision algorithms that are platform and hardware independent were firstly tested on a generic laptop with built in webcam running Microsoft Windows, mainly because it was easier for user interaction and parameter tuning.

\section{Software interfaces}

\subsection{Embedded operating system}

Ubuntu Linux distribution version 14.04.1 LTS was used on the platform, installed directly from the file system image provided by NVIDIA. Linux is great for this project because it is fully configurable, so that operating system overheads can be reduced to minimum by disabling unused services and even the graphical desktop environment. Also most Linux operations could be done through just command line interface, perhaps via a SSH shell access, therefore programming and control the platform could be done anywhere with internet connection, which is very convenient.

\subsection{Computer vision API}

OpenCV \cite{opencv} was used to implement the algorithms, due to its cross platform adaptability, easy to use and large number of existing algorithms ready for use and investigation. Furthermore, the OpenCV library for Jetson platform developed by NVIDIA was further optimised, can provide 2x-5x speed up compare to regular OpenCV \cite{NVIDIA:perf}. The OpenCV GPU module based on NVIDIA CUDA was also available, can provide 5x-20x speed up. These optimisations can reduce computation time dramatically, thus lower the power consumption further.

%% ----------------------------------------------------------------
%% Implement.tex
%% ----------------------------------------------------------------
\chapter{Implementation}

There were lots of different algorithms existing for object detection and tracking. Some of those algorithms were investigated in this project in order to identify a set of suitable algorithms that were both accurate and efficient enough to analysis video stream from camera in real-time.

\section{Background subtraction}

Being able to detect objects in a video frame is the first, also the most difficult and important step to do object tracking. This is generally accomplished by separation of foreground objects and background image. Three different object detection methodologies were investigated in this project.

\subsection{Colour based}

Colour can provides enough information of a specific object. For easier analysis of colour information, a hue-saturation-value (HSV) colourspace \cite[p.~301]{colourspace} representation converted from the original RGB colourspace is usually used, because it would be easier to filter a range of colour based on hue, saturation and brightness.

A simple colour based foreground mask can be generated easily by filtering target colour. An simple implementation \cite{MOTBOC.git} of colour filtering object detection algorithm was investigated, as shown in \fref{Figure:MOTBOC}.

\begin{figure}[H]
  \centering
  \includegraphics[width=0.6\columnwidth]{MOTBOC}
  \caption{Multi Object Tracking Based on Color (adapted from \cite{MOTBOC.git})}
  \label{Figure:MOTBOC}
\end{figure}

This implementation doesn't require a lot of computation, thus was very fast, could be suitable for robots that are tracking sonething like a single coloured ball or piece of paper, also could be useful for line racing car projects.

However, this implementation is very limited, it can only detects objects with single colour, cannot distinguishes the objects from similar background colour, relies heavily on manually adjusted colour threshold values, and is very sensitive to the variations of colourspaces from different cameras, not very adaptable. A complex environment may also results into lots of undesired detections, as shown in \fref{Figure:MOTBOC_F}.

\begin{figure}[H]
  \centering
  \includegraphics[width=0.6\columnwidth]{"MOTBOC failure"}
  \caption{Simple Multi Object Tracking Based on Color \cite{MOTBOC.git} at a complex environment}
  \label{Figure:MOTBOC_F}
\end{figure}

\subsection{Shape based}

Another important information about an object is its shape. By extracting hard object edges in the scene than apply appropriate shape transformation and filtering algorithms, an object could also be detected based on the shape.

\fref{Figure:circles} shows the image processed by circle detection, based on OpenCV's implementation of Hough Circle Transform \cite{opencv:hough_circle}. The coin at the top right corner had not been detected, because it actually appears to be a eclipse to the algorithm because of visual perspective.

\begin{figure}[H]
  \centering
  \subfigure [] {
    \includegraphics[width=0.45\columnwidth]{simple_original}
    \label{Figure:edges:original}
  }
  \subfigure [] {
    \includegraphics[width=0.45\columnwidth]{simple_blur}
    \label{Figure:edges:blur}
  }
  \subfigure [] {
    \includegraphics[width=0.45\columnwidth]{simple_edges}
    \label{Figure:edges:edges}
  }
  \subfigure [] {
    \includegraphics[width=0.45\columnwidth]{simple_circles}
    \label{Figure:edges:circles}
  }
  \caption{Circle detection. \subref{Figure:edges:original} The original image, \subref{Figure:edges:blur} image converted to gray scale and blurred, \subref{Figure:edges:edges} edges detected, \subref{Figure:edges:circles} circles detected}
  \label{Figure:circles}
\end{figure}

\subsection{Cascade Classifier}

Cascade classifier \cite{cascade} is another widely used technique for object detection. It concatenates several classifiers detecting different object features to recognise objects, and it can be trained both positively and negatively to improve accuracy. It was usually used for object classify, for example recognise human and different classes of vehicles in a single frame.

The OpenCV's cascade classifier implementation \cite{opencv:cc} of face and eye detection was investigated as shown in \fref{Figure:cc_face}.

\begin{figure}[H]
  \centering
  \includegraphics[width=0.6\columnwidth]{"CC face"}
  \caption{Face and eye detection cascade classifiers, detected face was circled by pink, whereas detected eyes were circled by blue}
  \label{Figure:cc_face}
\end{figure}

Noticeable frame rate drop and latency was experienced when using the cascade classifier implementation on the testing platform, suggests it was not a fast enough algorithm for real-time object tracking application. In addition, multiple classifier definition files were required for detecting different kinds of objects, or even different perspectives of the same object, which would require lots of computations.

\subsection{Motion based}

By differenceing current frame and previous frames, it is also possible to detects moving objects efficiently. The BGSLibrary \cite{bgslibrary} is specifically developed for this purpose, it offers 37 different background substraction algorithms using OpenCV, published under GNU GPL v3 license.

The article \cite{bgs:article} reviewed the algorithms available in the BGSLibrary, ranked 5 algorithms as the best methods for accuracy. Except the Pixel-Based Adaptive Segmenter (PBAS) algorithm which was removed due to patent issues, the other 4 algorithms listed in \tref{Table:bgs} were investigated in this project.

\begin{table}[H]
  \centering
  \begin{tabular}{cc}
  \toprule
  \textbf{Method ID} & \textbf{Method name}\\
  \midrule
  MultiLayerBGS & Multi-Layer BGS \\
  MixtureOfGaussianV1BGS & Gaussian Mixture Model \\
  LBAdaptiveSOM & Adaptive SOM \\
  DPWrenGABGS & Gaussian Average \\
  \bottomrule
  \end{tabular}
  \caption{Background substraction algorithms investigated (adapted from \cite{bgslibrary})}
  \label{Table:bgs}
\end{table}

\fref{Figure:bgs_frame} shows the foreground masks obtained from those 4 algorithms through 2 sample frame sequences available with the BGSLibrary \cite{bgslibrary}.

\begin{figure}[H]
  \centering
  \includegraphics[width=0.24\columnwidth]{bgs_frame/MultiLayerBGS/input}
  \includegraphics[width=0.24\columnwidth]{bgs_frame/MultiLayerBGS/mask}
  %\includegraphics[width=0.32\columnwidth]{bgs_frame/MultiLayerBGS/bkgmodel}
  \includegraphics[width=0.24\columnwidth]{bgs_video/MultiLayerBGS/input}
  \includegraphics[width=0.24\columnwidth]{bgs_video/MultiLayerBGS/mask}

  \includegraphics[width=0.24\columnwidth]{bgs_frame/MixtureOfGaussianV1BGS/input}
  \includegraphics[width=0.24\columnwidth]{bgs_frame/MixtureOfGaussianV1BGS/mask}
  \includegraphics[width=0.24\columnwidth]{bgs_video/MixtureOfGaussianV1BGS/input}
  \includegraphics[width=0.24\columnwidth]{bgs_video/MixtureOfGaussianV1BGS/mask}
  %\includegraphics[width=0.32\columnwidth]{na}

  \includegraphics[width=0.24\columnwidth]{bgs_frame/LBAdaptiveSOM/input}
  \includegraphics[width=0.24\columnwidth]{bgs_frame/LBAdaptiveSOM/mask}
  \includegraphics[width=0.24\columnwidth]{bgs_video/LBAdaptiveSOM/input}
  \includegraphics[width=0.24\columnwidth]{bgs_video/LBAdaptiveSOM/mask}
  %\includegraphics[width=0.32\columnwidth]{na}

  \includegraphics[width=0.24\columnwidth]{bgs_frame/DPWrenGABGS/input}
  \includegraphics[width=0.24\columnwidth]{bgs_frame/DPWrenGABGS/mask}
  \includegraphics[width=0.24\columnwidth]{bgs_video/DPWrenGABGS/input}
  \includegraphics[width=0.24\columnwidth]{bgs_video/DPWrenGABGS/mask}
  %\includegraphics[width=0.32\columnwidth]{na}
  \caption{Results obtained from background substraction algorithms. From left to right column: input sample 1, foreground mask obtained, input sample 2, foreground mask obtained. From top to bottom row: MultiLayerBGS, MixtureOfGaussianV1BGS, LBAdaptiveSOM and DPWrenGABGS.}
  \label{Figure:bgs_frame}
\end{figure}

It can be seen from \fref{Figure:bgs_frame} that MultiLayerBGS gave the best foreground masks, but it was also the slowest algorithm on the testing platform.

\section{Movement tracking}

After obtained the foreground object mask, they need to be interpreted as objects, then the object can be detected and tracked based on its position.

\subsection{Connected component analysis} \label{blob}

Connected component analysis, or Connected component labeling, is used for detecting connected regions (blobs). A blob detector can be used to mark and labeling individual objects from the foreground mask, therefore obtain parameters such as size, position and orientation of the object. Afterwards, by comparing nearby objects from previous frames, the objects can be tracked, and movement parameter such as velocity and acceleration can then be obtained by physical modelling.

There were also lots of free and open source blob detection libraries available, e.g. the simple blob detector came with OpenCV \cite{opencv:blob}, cvBlob library \cite{cvblob}.

This was not yet implemented at the time this progress report was written.

\subsection{Continuously Adaptive Meanshift}

Continuously Adaptive Meanshift (CAMshift) \cite{bradski1998computer} is a technique used to track a region of interest (ROI) in continuous frame sequences. It is based on Meanshift algorithm, which only track a fixed size ROI window, whereas CAMshift can handle target resize and rotation. In order to determine the ROI for CAMshift, the blob detector as described in Section \ref{blob} can also be used. This algorithm can also be easily implemented using OpenCV \cite{opencv:camshift}, and was used by lots of researches such as \cite{chu2007object}, \cite{xu2012moving} and \cite{nouar2006improved}.

This was not yet implemented at the time this progress report was written.

%\chapter{Project management}

\section{Risk assessment}

\section{Time management}

The topic of project changed once during week 7 in semester 1. The previous topic (\aref{Appendix:brief_prev}) was about reducing power consumption for algorithms that are utilising General Purpose GPU (GPGPU) technique, by adaptively controlling calculation precision. The NVIDIA Jetson TK1 development platform was decided to be used in this project. However, after investigated into the topic and experimented with some algorithms, the purposed precision control did not contribute significant impact on computation quality. Not much power saving would be possible to be made.

Therefore, I decided to change the project topic. The current project topic (\aref{Appendix:brief}) was quickly determined. The same hardware platform was still used, so that the time invested in the previous project was not wasted. Therefore, I was able to quickly catch up with the progress.

The Gantt chart was also changed due to project topic changing, as shown in \aref{Appendix:gantt}. The time invested for previous project topic was treated as background reading.
\todo{GANTT CHART UPDATE}

\section{Backing up}

%Source code management.
%Documents and materials.

The git version control system \cite{git} was used in this project. It has the feature that allows multiple users collaborate on the same repository. However, this is an individual project, git was used purely for backing up source code version history and synchronisation between multiple devices. Well-known git hosting website GitHub \cite{github} and university's SourceKettle server \cite{sourcekettle} were used for multiple backups.

All documents, background literals and materials used by the project were backed up using Microsoft's OneDrive service.

%\section{Remaining work}

%\begin{itemize}
%  \item Movement tracking based on blob detection (Section \ref{bg:tracking}, \ref{impl:tracking}).
%  \item A camera module need to be ordered and interfaced.
%  \item Automatic frame rate reduction based on tracking result.
%  \item Hardware downsampling and cropping based on ROI.
%  \item If time allows, energy-efficient object recognition may be implemented.
%\end{itemize}

\appendix
%\include{AppendixA}

\backmatter
\bibliography{Reference}
\end{document}
%% ----------------------------------------------------------------
