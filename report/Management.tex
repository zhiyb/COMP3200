\chapter{Project management}

\section{Risk assessment}

\section{Time management}

The topic of project changed once during week 7. The previous topic (\aref{Appendix:brief_prev}) was about reducing power consumption for algorithms that are utilising General Purpose GPU (GPGPU) technique for large scale parallelism floating point calculations, by dynamically controlling calculation precision. But after further investigation into the topic and experimented with some sample algorithms included with NVIDIA CUDA Toolkit, the precision control did not contribute significant impact on computation quality, made the project a bit pointless and not encouraging.

Therefore, the current project topic (\aref{Appendix:brief}) was quickly determined, still using the same development platform NVIDIA Tegra K1, so that the time invested for the previous project was not wasted that much. Therefore, sensible progress had achieved before this progress report was written.

The Gantt chart also changed due to project topic changing, as shown in \aref{Appendix:gantt}. The time invested for previous project topic was treated as background reading.

\section{Backing up}

Source code management.
Documents and materials.

The git version control system \cite{git} was used in this project. It has the feature that allows multiple users collaborate on the same repository, but since this is an individual project, git was mainly used for keeping code history and synchronise between multiple machines.

%\section{Remaining work}

%\begin{itemize}
%  \item Movement tracking based on blob detection (Section \ref{bg:tracking}, \ref{impl:tracking}).
%  \item A camera module need to be ordered and interfaced.
%  \item Automatic frame rate reduction based on tracking result.
%  \item Hardware downsampling and cropping based on ROI.
%  \item If time allows, energy-efficient object recognition may be implemented.
%\end{itemize}
