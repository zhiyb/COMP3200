\chapter{Conclusion} \label{Chapter:Conclusion}

\section{Achievements}

\todo{Power saving achieved.
Compare back to aims, specifications. (Why didn't?)}

\section{Constraints}

The algorithms used can only run at around 30 FPS maximum with $320 \times 240$ resolution. Objects moves faster than this, such as a basket ball may need a faster frame rate to be tracked. The maximum speed is also highly depending on the object to camera distance.

The algorithms implemented can only works with a steady background, limiting the application area to still camera application such as video surveillance.

\section{Further work}

\subsection{Camera driver}

The camera driver in collaboration with NVIDIA's MIPI-CSI driver is not stable enough. In case the application that are using the camera being forced stopped by the user or due to a programming bug, the drivers may crash the kernel due to lost synchronisation. This need further investigation.

When a buffer been marked as filled by the MIPI-CSI driver, it is actually still receiving data from the camera. Possible reasons are losing or misinterpreting synchronisation packets, camera configuration problem and bugs in MIPI-CSI driver implementation. A workaround was used, by receiving 2 filled buffers from the driver, then use the first one as valid data. However, this workaround gives 2 frames latency regardless of frame rate. This need further investigation.

The driver interface for controlling the frame rate, configuring exposure settings etc. need to be completed and standardised.

The Tegra K1 CPU has a built-in ISP that can convert Bayer pattern to standard RGB format without using CPU or GPU time. However, the documentation of this feature is confidential. The NVIDIA's video interface driver that are using this ISP is also undocumented in public domain. Power consumption may be reduced further if the ISP can be used.

\subsection{Application}

The application structure may be made more efficient.

The application can be a lot easier to use and configure with a better user interface.

Transmission of video stream and tracking results through network for Internet-of-Things applications may be useful to implement.

\subsection{Algorithms}

Further optimisations to the algorithms are possible. More efficient implementations that supports better resolution and faster frame rate are desired.

Blob tracking algorithm based on optical flow tracking results may be implemented.

More algorithms may be investigated. Especially algorithms that handle background movement, which are very useful in automotive applications.

\section{Evaluation}

\todo{Difficulties, solutions etc.}
